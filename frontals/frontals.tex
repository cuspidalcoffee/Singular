% Load the kaohandt class (with the default options)
\documentclass[
	%fontsize=10pt,
	%twoside=false,
	%secnumdepth=2,
	%abstract=true,
]{kaohandt}

\usepackage[english]{babel}
\usepackage[english=british]{csquotes}

%\usepackage{showframe}
%\usepackage{showlabels}

\usepackage{algebra}
\usepackage{fonts}
\usepackage{topology}
\usepackage{kaotheorems}

\usepackage{kaobiblio}
\addbibresource{report-template.bib} % Bibliography file

\usepackage{kaorefs}

%----------------------------------------------------------------------------------------

\begin{document}

%----------------------------------------------------------------------------------------
%	REPORT INFORMATION
%----------------------------------------------------------------------------------------

\title[The \texttt{frontals.lib} library]{The \texttt{frontals.lib} library}

\author[CMC]{C. Muñoz-Cabello}

\date{\today}

%----------------------------------------------------------------------------------------
%	TITLE AND ABSTRACT
%----------------------------------------------------------------------------------------

\maketitle

\margintoc

\begin{abstract}
\noindent
\end{abstract}

% {\noindent\textbf{Keywords:} \LaTeX, Kao, handout, article, report}

\medskip

%----------------------------------------------------------------------------------------
%	MAIN BODY
%----------------------------------------------------------------------------------------

\section{Introduction}

Given a corank $1$ analytic map $f\colon (\mb{K}^n,0) \to (\mb{K}^{n+1},0)$, we may always choose coordinates in the source and target such that
\begin{equation}\label{corank 1}
	f(x,y)=(x, p(x,y), q(x,y))
\end{equation}
for some $p,q \in \mf{m}_n^2$, $x \in \mb{K}^n$, $y \in \mb{K}$.
Nuño-Ballesteros showed in \sidecite{Nuno_CuspsAndNodes} that, under this parametrisation, $f$ is a frontal if and only if $p_y | q_y$ or $q_y | p_y$.
We shall say that $f$ is in \emph{prenormal form} if $p_y|q_y$, in which case we write $\mu=q_y/p_y$.

% \blindtext\sidenote[][*-8]{\blindtext}

\section{Methods}

\section{Frontalise}
If we assume that $f$ is analytic and given as in Equation \eqref{corank 1} (not necessarily being frontal), we may write
	\begin{align*}
		p(x,y)&=p_2(x)y^2+p_3(x)y^3+\dots\\
		q(x,y)&=q_2(x)y^2+q_3(x)y^3+\dots
	\end{align*}
in which case $p_y|q_y$ if and only if there are functions $u_1,\dots, u_{k-1} \in \mb{K}\{x\}$ such that
	\[kq_k=kp_ku_0+(k-1)p_{k-1}u_1+\dots+p_1u_{k-1}\]
This gives a method to generate examples of frontal maps using analytic plane curves.

It is shown in \sidecite{MNO_FrontalDeformations} that, if $f$ is the versal deformation of some analytic plane curve $\gamma$, there exists a graph-type immersion $h$ such that the pull-back $h^*f$ is a versal frontal unfolding of $\gamma$.

\section{Getnormal}
A smooth map $f\colon (\mb{K}^n,S) \to (\mb{K}^{n+1},0)$ is frontal if and only if there is a germ of differential $1$-form $\nu$ on $(\mb{K}^{n+1},0)$ that does not vanish on $f(S)$ and such that
	\[\nu(df \circ \xi)=0\]
for all $\xi \in \theta_n$.
The differential form $\nu$ represents a line that is normal to the hypersurface $f(\mb{K}^n,S)$ at every point, and is usually computed as the orthogonal complement of $df_x(T_x\mb{K}^n)$ for all regular values $x \in (\mb{K}^n,S)$ of $f$.
Assuming $f$ is a proper frontal (i.e. the germ of $\mathrm{Sing}(f)$ at $S$ is nowhere dense in $(\mb{K}^n,S)$) implies that $\nu$ is uniquely determined up to a constant factor.

Orthogonalising an $n\times p$ matrix using the Gram-Schmidt process has a computational complexity of $\mc{O}(np^2)$ flops\sidenote{https://stackoverflow.com/a/27986858},
However, since we are only interested in the last column of the matrix, it is more efficient to perform textbook orthogonalisation (i.e. solving the corresponding system of equations).
It is known that Gauss echelon reduction using the LU decomposition has a computational complexity of $\mc{O}(n^2)$ flops.

If $f$ is given in prenormal form, a simple computation shows that
\begin{align*}
	\nu_j=\frac{\p q}{\p x_j}-\mu\frac{\p p}{\p x_j}; && \nu_n=\mu; && \nu_{n+1}=-1
\end{align*}
gives one such $\nu$, with a computational complexity of $\mc{O}(n)$ (assuming partial derivates are known).

\section{Invariants}
\sidecite{MNO_FrontalSurfaces}

\section{IsFrontal}
\sidecite{Ishikawa_Survey}

\section{wfinvariants}
\sidecite{MMR_ZeroSchemes}

\appendix % From here onwards, chapters are numbered with letters, as is the appendix convention

\section{Appendix}

\section{Prenormal}

% \blindtext

%----------------------------------------------------------------------------------------
%	BIBLIOGRAPHY
%----------------------------------------------------------------------------------------

% The bibliography needs to be compiled with biber using your LaTeX editor, or on the command line with 'biber main' from the template directory

\printbibliography[title=Bibliography] % Set the title of the bibliography and print the references

\end{document}
